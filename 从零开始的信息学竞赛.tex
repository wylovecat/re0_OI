\documentclass[lang=cn,10pt]{OIBooks}

\title{从零开始的信息学竞赛}
\subtitle{一名野生教练的教学笔记}
\author{吴尧}
%\institute{Elegant\LaTeX{} Program}
\date{\today}
\version{0.1}
\bioinfo{B站}{爱学习的咸鱼君}

\extrainfo{古之立大事者,不惟有超世之才,亦必有坚忍不拔之志。—— 苏轼}

\setcounter{tocdepth}{3}

\logo{neo.jpg}
\cover{7.jpg}

% 本文档命令

\usepackage{longtable}
\usepackage{array}
\newcommand{\ccr}[1]{\makecell{{\color{#1}\rule{1cm}{1cm}}}}

%修改标题页的橙色带
\definecolor{customcolor}{RGB}{32,178,170}
\colorlet{coverlinecolor}{customcolor}

\begin{document}

\maketitle%封面
\frontmatter
\thispagestyle{empty}
\begin{center}
  \textbf{\LARGE 文档概述}
\end{center}

\section*{排版模板参考}
\subsection*{Elegant\LaTeX{} 系列模板 \md{[核心版本]}}
ElegantLATEX 项目组致力于打造一系列美观、优雅、简便的模板方便用户使用。 目前由
ElegantNote,ElegantBook,ElegantPaper 组成,分别用于排版笔记,书籍和工作论文。
\begin{itemize}
  \item 官网:\href{https://elegantlatex.org/}{https://elegantlatex.org/}
  \item GitHub 网址:\href{https://github.com/ElegantLaTeX/}{https://github.com/ElegantLaTeX/}
\end{itemize}
\section*{书籍内容参考}
\subsection*{《算法竞赛》 \md{[罗勇军]}}
本书是一本全面、深入解析与算法竞赛有关的数据结构、算法、代码的计算机教材

本书包括十个专题: 基础数据结构、基本算法、搜索、高级数据结构、动态规划、数论和线性代数、组合数学、计算几何、字符串和图论。本书覆盖了绝大多数算法竞赛考点。

本书解析了算法竞赛考核的数据结构、算法; 组织了每个知识点的理论解析和经典例题; 给出了简洁、精要的模板代码; 通过明快清晰的文字、透彻的图解,实现了较好的易读性。

本书的读者对象是参加算法竞赛的中学生和大学生、准备面试IT企业算法题的求职者、需要提高算法能力的开发人员,以及对计算机算法有兴趣的广大科技工作者。
\subsection*{《深入浅出程序设计竞赛 基础篇》 \md{[汪楚奇]}}
本书分为4部分:第1部分介绍C++语言的基础知识,包括表达式、变量、分支、循环、数组、函数、字符串、结构体等内容;第2部分介绍一些基础算法,包括模拟、高精度、排序、枚举、递推、递归、贪心、二分、搜索等;第3部分介绍几种简单常用的数据结构,包括线性表、二叉树、并查集、哈希表和图;第4部分是在算法竞赛中需要使用的数学基础,包括位运算与进制转换、计数原理、排列与组合、质数与合数、约数与倍数等概念。

本书主要面向从未接触过程序设计竞赛(包括NOI系列比赛、ICPC系列比赛)的选手,也适用于稍有接触算法、希望进一步巩固算法基础的读者。

本书提供一些在线的配套资源,例如课件或勘误表,读者可以发邮件至编辑邮箱1548103297@qq.com索取。
\subsection*{《深入浅出程序设计竞赛 进阶篇》\md{[汪楚奇]}}
该书未出版,可通过购买洛谷月赛年卡,以课程赠品的形式获得书稿。

本书分为5部分:第1部分介绍 进阶技巧与思想,包括常见优化技巧、前缀和、差分与离散化、分治与倍增等内容;第2部分介绍一些进阶数据结构,包括二叉堆与树状数组、线段树、字符串等内容;第3部分介绍图论相关算法,包括树、最短路、最小生成树、连通性等内容;第4部分介绍动态规划相关知识,包括线性动态规划、区间与环形动态规划、树与图上的动态规划等内容;第5部分介绍数学,包括进阶数论、组合数学与计算、概率与统计等内容。

\subsection*{《算法竞赛入门经典》 \md{[刘汝佳]}}
《算法竞赛入门经典(第2版)》是一本算法竞赛的入门与提高教材,把C/C++语言、算法和解题有机地结合在一起,淡化理论,注重学习方法和实践技巧。全书内容分为12章,包括程序设计入门、循环结构程序设计、数组和字符串、函数和递归、C++与STL入门、数据结构基础、暴力求解法、高效算法设计、动态规划初步、数学概念与方法、图论模型与算法、高级专题等内容,覆盖了算法竞赛入门和提高所需的主要知识点,并含有大量例题和习题。书中的代码规范、简洁、易懂,不仅能帮助读者理解算法原理,还能教会读者很多实用的编程技巧;书中包含的各种开发、测试和调试技巧也是传统的语言、算法类书籍中难以见到的。

《算法竞赛入门经典(第2版)》可作为全国青少年信息学奥林匹克联赛(NOIP)复赛教材、全国青少年信息学奥林匹克竞赛(NOI)和ACM国际大学生程序设计竞赛(ACM/ICPC)的训练资料。
\subsection*{《算法竞赛进阶指南》 \md{[李煜东]}}


本书主要根据CCF-NOI信息学奥林匹克竞赛涉及的知识体系进行编写,对计算机程序设计的基本技能——数据结构与算法进行了深入的讲解。

本书面向已经掌握至少一门程序设计语言、对于算法设计有入门性认识的读者,以各类知识点之间的贯穿联系为主线,通过各种模型与例题对各种思维方向进行深入引导,让读者在阅读本书后对算法设计初步具有整体掌控性的理解。能够让读者由浅入深地体会算法,学习算法。

本书融合了作者在算法设计教育领域、算法竞赛参赛与指导领域10年来的一线经验,其特色是训练读者算法设计的思维习惯,而非对知识流水的记忆性诵读,能让认真阅读本书并完成所有练习的读者,逐渐具有NOIP竞赛一等奖以上的实力。

\subsection*{《算法训练营 进阶篇》\md{[陈小玉]}}
《算法训练营:海量图解+竞赛刷题(进阶篇)》以海量图解的形式,详细讲解常用的数据结构与算法,并结合竞赛实例引导读者进行刷题实战。通过对本书的学习,读者可掌握22种高级数据结构、7种动态规划算法、5种动态规划优化技巧,以及5种网络流算法,并熟练应用各种算法解决实际问题。

\newpage
\frontmatter
\thispagestyle{empty}

\begin{longtable}{m{3cm}<{\centering}m{5cm}<{\centering}m{3cm}<{\centering}}
	\caption*{{\LARGE 致谢名单}} \\ \toprule
	编号 & 昵称  & 打赏 \\ \midrule
	null & null & null  \\
	null & null & null \\ 
	\bottomrule
\end{longtable}

\begin{center}	
	如果你喜欢本文档,欢迎打赏! \\[1pt]
	\begin{tabular}{ccc}
		\includegraphics[width=.2\linewidth]{00chapter/wxpng}   & \qquad  \qquad &
		\includegraphics[width=.2\linewidth]{00chapter/zfbpng}  
	\end{tabular}
\end{center}


\tableofcontents%目录

\mainmatter
\part{语言基础篇}{语言基础篇}
\chapter{粮草先行}
\begin{introduction}
	\item 计算机选购
	\item 操作系统
	\item \texttt{IDE}安装
	\item 网站推荐
	\item 学习方法
	\item 学习心态
\end{introduction}

俗话说三军未动,粮草先行。在正式开启信奥的学习之前,我们先把准备工作做好。
\section{硬件环境准备}
首先,信奥学习是需要动手编程的,那么一台电脑是必不可少的。简单说下电脑选购的事情。若我们的电脑只用来考虑信奥学习,完全不去考虑游戏的事的话,不夸张地说,只要是台能运行起来的电脑,其实都是可以拿来用的。继承下家长淘汰下来的电脑是最省钱的方案。而如果说家里还没有电脑,要买个新的话,建议是不要去线下电脑城购买,去了$ 90\% $是被宰一顿。推荐在京东购买电脑。

而对于是购买台式机还是购买笔记本,则是看你有没有携带的需求,如果你不需要带着电脑到处跑的话,同等价位下,台式机的性能是要超过笔记本的。

对于配置的选购的话,信奥学习这块,不用去刻意追求显卡的好坏,CPU自带的核显是完全够用的,重点关注下CPU、内存和硬盘即可。
\section{软件环境准备}
\subsection{操作系统}
操作系统这块,虽然在复赛的时候,是要求在\texttt{NOI Linux 2.0}\footnote{Ubuntu 20.04的魔改版本,封装了竞赛常用的一些软件,并阉割了一部分东西。}系统上进行的,但是从初学者上手的角度来说,建议可以从Windows操作系统开始,Window10/11均可。后期,等我们对计算机的相关操作已经比较熟练了,我们再转战到\texttt{Linux}平台,计划是到算法学习阶段,我们会在\texttt{NOI Linux 2.0}下进行相关内容的学习。
\subsection{集成开发环境}
集成开发环境(Integrated Development Environment,IDE),是用于我们写C++程序的。Windows上推荐两个,一个是\texttt{Dev-C++},另一个是\texttt{Code::Blocks}。两者都能开箱即用。\texttt{Dev-C++}比较小巧,可快速地进行单文件编译运行,上手门槛较低。而\texttt{Code::Blocks}不仅仅\texttt{Windows}平台下有,在\texttt{Noi Linux 2.0} 中也有该编辑器,能更顺利的进行学习环境的迁移。

两个IDE任选一个即可,重点还是在语言与算法学习上。

\subsubsection{\texttt{Dev-C++}安装}
\textbf{软件下载地址} :\href{https://wloving.lanzouq.com/dev-cpp}{https://wloving.lanzouq.com/dev-cpp}

下载好之后双击\texttt{exe}文件。
\begin{figure}[H]
\centering
\includegraphics[width=0.6\linewidth]{01chapter/img/dev安装01}
\caption{双击安装包}
\label{fig:dev01}
\end{figure}

双击后,等待程序提取安装包内容。
\begin{figure}[H]
\centering
\includegraphics[width=0.6\linewidth]{01chapter/img/dev安装02}
\caption{等待提取安装包}
\label{fig:dev02}
\end{figure}
点击\texttt{I Agree},同意相关事项。
\begin{figure}[H]
\centering
\includegraphics[width=0.6\linewidth]{01chapter/img/dev安装03}
\caption{点击同意}
\label{fig:dev03}
\end{figure}
在安装组件部分,选择默认的就行,直接点击\texttt{next}即可。
\begin{figure}[H]
\centering
\includegraphics[width=0.6\linewidth]{01chapter/img/dev安装04}
\caption{安装组件}
\label{fig:dev04}
\end{figure}
对于安装位置,也是默认即可。
\begin{figure}[H]
\centering
\includegraphics[width=0.6\linewidth]{01chapter/img/dev安装05}
\caption{安装位置}
\label{fig:dev05}
\end{figure}

\begin{figure}[H]
\centering
\includegraphics[width=0.6\linewidth]{01chapter/img/dev安装06}
\caption{安装过程}
\label{fig:dev06}
\end{figure}
安装结束,点击\texttt{Finish}。
\begin{figure}[H]
\centering
\includegraphics[width=0.6\linewidth]{01chapter/img/dev安装07}
\caption{安装结束}
\label{fig:dev07}
\end{figure}
第一次运行时,会出现语言选择部分,我们选择简体中文。
\begin{figure}[H]
\centering
\includegraphics[width=0.6\linewidth]{01chapter/img/dev安装08}
\caption{语言选择}
\label{fig:dev08}
\end{figure}
主题部分,选默认的即可。
\begin{figure}[H]
\centering
\includegraphics[width=0.6\linewidth]{01chapter/img/dev安装09}
\caption{主题选择}
\label{fig:dev09}
\end{figure}
安装完成后,我们测试下软件,看是否能正常编译运行C++程序。首先,先在软件左上角点击\texttt{文件-新建-源文件},或者是通过快捷键\texttt{Ctrl+N}的方式进行新建。
\begin{figure}[H]
\centering
\includegraphics[width=0.6\linewidth]{01chapter/img/dev安装10}
\caption{新建源文件}
\label{fig:dev10}
\end{figure}
复制以下代码以测试程序是否能正常工作,复制粘贴完毕后,点击上方猜测方块或者是快捷键\texttt{F11}。选择好保存位置后即可编译运行。


\begin{minted}{C++}
#include <iostream>
using namespace std;
int main()
{
	cout<<"Hello world";
	return 0;
}
\end{minted}

\begin{figure}[H]
\centering
\includegraphics[width=0.6\linewidth]{01chapter/img/dev安装11}
\caption{编译运行}
\label{fig:dev11}
\end{figure}

\begin{figure}[H]
\centering
\includegraphics[width=0.6\linewidth]{01chapter/img/dev安装12}
\caption{运行结果}
\label{fig:dev12}
\end{figure}

\subsubsection{\texttt{Code::Blocks}安装}

\subsubsection{网站推荐}
推荐几个对之后的信奥大有帮助的网站,可以保存在浏览器的收藏夹中。
\begin{itemize}
\item \texttt{NOI}官网:\href{https://www.noi.cn}{https://www.noi.cn}
\item 洛谷:\href{https://www.luogu.com.cn}{https://www.luogu.com.cn}
\end{itemize}

\section{心态与方法}
“纸上得来终觉浅,绝知此事要躬行。”要将书上、课堂上的知识转化为自己的能力,需要经过大量的练习。在学习的过程中一定要注重上机实操与独立思考。对于学到的知识需要时常复习、总结。建议养成写学习博客的习惯,用自己的语言记录学习的内容。





\printbibliography[heading=bibintoc, title=\ebibname]
\appendix

\chapter{基本数学工具}


本附录包括了计量经济学中用到的一些基本数学,我们扼要论述了求和算子的各种性质,研究了线性和某些非线性方程的性质,并复习了比例和百分数。我们还介绍了一些在应用计量经济学中常见的特殊函数,包括二次函数和自然对数,前 4 节只要求基本的代数技巧,第 5 节则对微分学进行了简要回顾;虽然要理解本书的大部分内容,微积分并非必需,但在一些章末附录和第 3 篇某些高深专题中,我们还是用到了微积分。

\section{求和算子与描述统计量}

\textbf{求和算子} 是用以表达多个数求和运算的一个缩略符号,它在统计学和计量经济学分析中扮演着重要作用。如果 $\{x_i: i=1, 2, \ldots, n\}$ 表示 $n$ 个数的一个序列,那么我们就把这 $n$ 个数的和写为:

\begin{equation}
\sum_{i=1}^n x_i \equiv x_1 + x_2 +\cdots + x_n
\end{equation}



\end{document}
