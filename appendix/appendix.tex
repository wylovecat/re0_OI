\chapter{粮草先行}
\begin{introduction}
	\item 计算机选购
	\item 操作系统
	\item \texttt{IDE}安装
	\item 网站推荐
	\item 学习方法
	\item 学习心态
\end{introduction}

俗话说三军未动,粮草先行。在正式开启信奥的学习之前,我们先把准备工作做好。
\section{硬件环境准备}
首先,信奥学习是需要动手编程的,那么一台电脑是必不可少的。简单说下电脑选购的事情。若我们的电脑只用来考虑信奥学习,完全不去考虑游戏的事的话,不夸张地说,只要是台能运行起来的电脑,其实都是可以拿来用的。继承下家长淘汰下来的电脑是最省钱的方案。而如果说家里还没有电脑,要买个新的话,建议是不要去线下电脑城购买,去了$ 90\% $是被宰一顿。推荐在京东购买电脑。

而对于是购买台式机还是购买笔记本,则是看你有没有携带的需求,如果你不需要带着电脑到处跑的话,同等价位下,台式机的性能是要超过笔记本的。

对于配置的选购的话,信奥学习这块,不用去刻意追求显卡的好坏,CPU自带的核显是完全够用的,重点关注下CPU、内存和硬盘即可。
\section{软件环境准备}
\subsection{操作系统}
操作系统这块,虽然在复赛的时候,是要求在\texttt{NOI Linux 2.0}\footnote{Ubuntu 20.04的魔改版本,封装了竞赛常用的一些软件,并阉割了一部分东西。}系统上进行的,但是如果你是零基础,从容易上手的角度来说,建议可以从Windows操作系统开始,Window10/11均可。后期,等我们对计算机的相关操作已经比较熟练了,我们再转战到\texttt{Linux}平台;而如果是有计算机基础的可以直接安装Linux系统进行相关学习。
\subsection{集成开发环境}
集成开发环境(Integrated Development Environment,IDE),是用于我们写C++程序的。CCF\footnote{中国计算机学会}在\texttt{NOI Linux 2.0}中已经预装了不少软件,包括\texttt{Code::Blocks}、\texttt{Geany}、和\texttt{VS Code}等。这三个软件都是跨平台的,都有 Windows 和 Linux 版本,可以选取一个软件进行安装使用,也方便不同操作系统之间的习惯迁移。

另外,一些线下的编程类比赛,主办方为了省事往往不会特意安装Linux 系统,往往是Windows + \texttt{Dev C++}的组合,所以也可以选择使用\texttt{Dev C++}或它的魔改强化版本小熊猫C++进行初期的学习,他们的优点是开箱即用,不用额外配置,可快速地进行单文件编译运行,上手门槛较低。

\texttt{Code::Blocks}也能快速进行单文件编译运行,但是界面是英文的,存在一定的门槛。好处是,不仅仅\texttt{Windows}平台下有,在\texttt{Noi Linux 2.0} 中也有该软件,能更顺利的进行学习环境的迁移。

对于\texttt{VS Code},使用需要一定的门槛,需要能把环境配置好。但是配置好的VS Code在软件使用的便捷性和颜值方面是强于其他软件的。

综上,对于计算机小白,建议是Windows + 小熊猫C++的组合进行语言的入门,后期等对电脑较为熟悉了,可转移到 Linux平台上,使用 \texttt{Code::Blocks}或\texttt{VS Code}进行编程。


