\frontmatter
\thispagestyle{empty}
\begin{center}
  \textbf{\LARGE 文档概述}
\end{center}

\section*{排版模板参考}
\subsection*{Elegant\LaTeX{} 系列模板 \md{[核心版本]}}
ElegantLATEX 项目组致力于打造一系列美观、优雅、简便的模板方便用户使用。 目前由
ElegantNote,ElegantBook,ElegantPaper 组成,分别用于排版笔记,书籍和工作论文。
\begin{itemize}
  \item 官网:\href{https://elegantlatex.org/}{https://elegantlatex.org/}
  \item GitHub 网址:\href{https://github.com/ElegantLaTeX/}{https://github.com/ElegantLaTeX/}
\end{itemize}
\section*{书籍内容参考}
\subsection*{《算法竞赛》 \md{[罗勇军]}}
本书是一本全面、深入解析与算法竞赛有关的数据结构、算法、代码的计算机教材

本书包括十个专题: 基础数据结构、基本算法、搜索、高级数据结构、动态规划、数论和线性代数、组合数学、计算几何、字符串和图论。本书覆盖了绝大多数算法竞赛考点。

本书解析了算法竞赛考核的数据结构、算法; 组织了每个知识点的理论解析和经典例题; 给出了简洁、精要的模板代码; 通过明快清晰的文字、透彻的图解,实现了较好的易读性。

本书的读者对象是参加算法竞赛的中学生和大学生、准备面试IT企业算法题的求职者、需要提高算法能力的开发人员,以及对计算机算法有兴趣的广大科技工作者。
\subsection*{《深入浅出程序设计竞赛 基础篇》 \md{[汪楚奇]}}
本书分为4部分:第1部分介绍C++语言的基础知识,包括表达式、变量、分支、循环、数组、函数、字符串、结构体等内容;第2部分介绍一些基础算法,包括模拟、高精度、排序、枚举、递推、递归、贪心、二分、搜索等;第3部分介绍几种简单常用的数据结构,包括线性表、二叉树、并查集、哈希表和图;第4部分是在算法竞赛中需要使用的数学基础,包括位运算与进制转换、计数原理、排列与组合、质数与合数、约数与倍数等概念。

本书主要面向从未接触过程序设计竞赛(包括NOI系列比赛、ICPC系列比赛)的选手,也适用于稍有接触算法、希望进一步巩固算法基础的读者。

本书提供一些在线的配套资源,例如课件或勘误表,读者可以发邮件至编辑邮箱1548103297@qq.com索取。
\subsection*{《深入浅出程序设计竞赛 进阶篇》\md{[汪楚奇]}}
该书未出版,可通过购买洛谷月赛年卡,以课程赠品的形式获得书稿。

本书分为5部分:第1部分介绍 进阶技巧与思想,包括常见优化技巧、前缀和、差分与离散化、分治与倍增等内容;第2部分介绍一些进阶数据结构,包括二叉堆与树状数组、线段树、字符串等内容;第3部分介绍图论相关算法,包括树、最短路、最小生成树、连通性等内容;第4部分介绍动态规划相关知识,包括线性动态规划、区间与环形动态规划、树与图上的动态规划等内容;第5部分介绍数学,包括进阶数论、组合数学与计算、概率与统计等内容。

\subsection*{《算法竞赛入门经典》 \md{[刘汝佳]}}
《算法竞赛入门经典(第2版)》是一本算法竞赛的入门与提高教材,把C/C++语言、算法和解题有机地结合在一起,淡化理论,注重学习方法和实践技巧。全书内容分为12章,包括程序设计入门、循环结构程序设计、数组和字符串、函数和递归、C++与STL入门、数据结构基础、暴力求解法、高效算法设计、动态规划初步、数学概念与方法、图论模型与算法、高级专题等内容,覆盖了算法竞赛入门和提高所需的主要知识点,并含有大量例题和习题。书中的代码规范、简洁、易懂,不仅能帮助读者理解算法原理,还能教会读者很多实用的编程技巧;书中包含的各种开发、测试和调试技巧也是传统的语言、算法类书籍中难以见到的。

《算法竞赛入门经典(第2版)》可作为全国青少年信息学奥林匹克联赛(NOIP)复赛教材、全国青少年信息学奥林匹克竞赛(NOI)和ACM国际大学生程序设计竞赛(ACM/ICPC)的训练资料。
\subsection*{《算法竞赛进阶指南》 \md{[李煜东]}}


本书主要根据CCF-NOI信息学奥林匹克竞赛涉及的知识体系进行编写,对计算机程序设计的基本技能——数据结构与算法进行了深入的讲解。

本书面向已经掌握至少一门程序设计语言、对于算法设计有入门性认识的读者,以各类知识点之间的贯穿联系为主线,通过各种模型与例题对各种思维方向进行深入引导,让读者在阅读本书后对算法设计初步具有整体掌控性的理解。能够让读者由浅入深地体会算法,学习算法。

本书融合了作者在算法设计教育领域、算法竞赛参赛与指导领域10年来的一线经验,其特色是训练读者算法设计的思维习惯,而非对知识流水的记忆性诵读,能让认真阅读本书并完成所有练习的读者,逐渐具有NOIP竞赛一等奖以上的实力。

\subsection*{《算法训练营 进阶篇》\md{[陈小玉]}}
《算法训练营:海量图解+竞赛刷题(进阶篇)》以海量图解的形式,详细讲解常用的数据结构与算法,并结合竞赛实例引导读者进行刷题实战。通过对本书的学习,读者可掌握22种高级数据结构、7种动态规划算法、5种动态规划优化技巧,以及5种网络流算法,并熟练应用各种算法解决实际问题。
